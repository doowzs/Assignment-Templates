\documentclass[10pt]{beamer}
\usepackage{amsmath, amssymb}

\usetheme[numbering=fraction, sectionpage=progressbar, block=fill]{metropolis}

\definecolor{teal}{rgb}{0, 0.5859, 0.4805}
\setbeamercolor{progress bar}{fg=teal}
\setbeamercolor{title separator}{fg=teal}   
\usepackage{appendixnumberbeamer}

\usepackage{booktabs}
\usepackage[scale=2]{ccicons}

\usepackage{pgfplots}
\usepgfplotslibrary{dateplot}

\usepackage{xspace}
\newcommand{\themename}{\textbf{\textsc{metropolis}}\xspace}
\usefonttheme[onlymath]{serif}

\usepackage[AutoFakeBold, AutoFakeSlant]{xeCJK}

\newcommand{\song}{\CJKfamily{song}} 
\newcommand{\hei}{\CJKfamily{hei}} 
\newcommand{\kai}{\CJKfamily{kai}} 
\newcommand{\fs}{\CJKfamily{fs}}

\usepackage{standalone}
\usepackage{booktabs, threeparttable}
\usepackage{multirow}

\usepackage{listings}
\lstset{
    language={[ISO]C++},
    commentstyle=\color{red!50!green!50!blue!50},  
    rulesepcolor=\color{red!20!green!20!blue!20},  
    keywordstyle=\color{blue!90}\bfseries,   
    showstringspaces=false, 
    stringstyle=\ttfamily,
} 

\usepackage{tikz}
\usetikzlibrary{
    arrows,
    calc,
    %fit,
    %patterns,
    %plotmarks,
    shapes.geometric,%
    shapes.misc,
    shapes.symbols,
    %shapes.arrows,
    %shapes.callouts,
    %shapes.multipart,
    %shapes.gates.logic.US,
    %shapes.gates.logic.IEC,
    %circuits.logic.US,
    %circuits.logic.IEC,
    %circuits.logic.CDH,
    %circuits.ee.IEC,
    %datavisualization,
    %datavisualization.formats.functions,
    %er,
    %automata,
    %backgrounds,
    %chains,
    %topaths,
    %trees,
    %petri,
    mindmap,
    %matrix,
    %calendar,
    %folding,
    %fadings,
    %shadings,
    %spy,
    %through,
    %turtle,
    positioning,%
    %scopes,
    %decorations.fractals,
    %decorations.shapes,
    %decorations.text,
    decorations.pathmorphing,
    decorations.pathreplacing,
    %decorations.footprints,
    %decorations.markings,
    %shadows,
    %lindenmayersystems,
    %intersections,
    %fixedpointarithmetic,
    %fpu,
    %svg.path,
    %external,
}%

%%%%%%%%%%%%%%%%%%%%%%%%%%%%%%%%%%%%%%%%%%%%%%%%%%%%%%%%%%%

%title
\title{Beamer Metropolis Mod README}
\author[Tianyun Zhang]{张天昀}
\institute[CS@NJU]{南京大学计算机科学与技术系\\ 171860508@smail.nju.edu.cn}
\renewcommand{\today}{\number\year 年\number\month 月\number\day 日}
\date{\today}

\renewcommand{\figurename}{图}
\renewcommand{\tablename}{表}


\begin{document}

\begin{frame}
  \maketitle
\end{frame}

\begin{frame}{说明}
    由于NJU beamer模板实在太丑了,于是我找了这个 3000+ Star 的 \href{https://github.com/matze/mtheme}{beamer 主题}来当我的新的beamer模板,顺便写一下beamer很多东西用法的memo。
    
    \begin{block}{LICENSE}
        本文档使用CC-BY-SA 4.0协议授权。
        \ccbysa
    \end{block}
\end{frame}

\begin{frame}{目录}
    \setbeamertemplate{section in toc}[sections numbered]
    \tableofcontents[hideallsubsections]
\end{frame}

\section{基础内容}
\begin{frame}[fragile]{页面布局}
    一般beamer总是从左往右、从上至下(居中)来分布页面内容,但是可以手动分栏。
    \begin{verbatim}
        \begin{columns}[T,onlytextwidth]
            \column{0.5\textwidth}
                Foo
            \column{0.5\textwidth}
                Bar
        \end{columns}
    \end{verbatim}
    就会产生如下效果:
    \begin{columns}[T,onlytextwidth]
        \column{0.5\textwidth}
        Foo
        \column{0.5\textwidth}
        Bar
    \end{columns}
\end{frame} 

\begin{frame}[fragile]{页面元素}
    页面元素有很多种,除了文字以外,有
    公式:$$y = x^2$$
    图片:\vspace{2mm} \\
    \begin{center}
        \includegraphics[scale=0.1]{demo.jpg}
    \end{center}
    
    表格:\vspace{2mm} \\
    \begin{center}
        \begin{tabular}{|c|c|c|}
            \hline
            \multicolumn{2}{|c|}{111} & \multirow{2}{*}{FooBar} \\
            \cline{1-2}
            Foo & Bar &  \\
            \hline
        \end{tabular}
    \end{center}
\end{frame} 

\begin{frame}{页面元素}
    tikz绘制的图表:
    \begin{figure}
        \begin{tikzpicture}
            \begin{axis}[
                mbarplot,
                xlabel={Foo},
                ylabel={Bar},
                width=0.9\textwidth,
                height=6cm,
            ]

            \addplot plot coordinates {(1, 20) (2, 25) (3, 22.4) (4, 12.4)};
            \addplot plot coordinates {(1, 18) (2, 24) (3, 23.5) (4, 13.2)};
            \addplot plot coordinates {(1, 10) (2, 19) (3, 25) (4, 15.2)};

            \legend{lorem, ipsum, dolor}

            \end{axis}
        \end{tikzpicture}
    \end{figure}
\end{frame}

\begin{frame}[fragile]{页面元素}
    代码块(页面需要fragile):
    \begin{verbatim}
        int main(){return 0;}
    \end{verbatim}
    引言:
    \begin{quote}
        同*恋才是真爱。——鲁迅
    \end{quote}
    \begin{quote}
        ‘The Kouga Ninja Scrolls’ is a historical fantasy novel about ninja written in 1958-1959 by the Japanese author Futaro Yamada. This is the first volume of the Ninja Scrolls series written by Yamada in 1958-2001. The book has been translated into English by Geoff Sant and was published by Del Rey in December 2006.
        \begin{flushright}
            -- from Wikipedia, the free encyclopedia
        \end{flushright}
    \end{quote}
\end{frame}

\begin{frame}[standout]
    \textsl{The end.}
\end{frame}
\end{document}