\usepackage{amsmath, amssymb}

\usepackage[AutoFakeBold, AutoFakeSlant]{xeCJK}

%\setCJKmainfont{SimSun}
%\setCJKmonofont{KaiTi}
%\setCJKsansfont{Microsoft YaHei} % or SimHei

\newcommand{\song}{\CJKfamily{song}} 
\newcommand{\hei}{\CJKfamily{hei}} 
\newcommand{\kai}{\CJKfamily{kai}} 
\newcommand{\fs}{\CJKfamily{fs}}

\usepackage{standalone}
\usepackage{booktabs, threeparttable}

\usepackage{listings}
\lstset{
	language={[ISO]C++},
    commentstyle=\color{red!50!green!50!blue!50},  
  	rulesepcolor=\color{red!20!green!20!blue!20},  
 	keywordstyle=\color{blue!90}\bfseries,   
  	showstringspaces=false, 
  	stringstyle=\ttfamily,
} 

\usepackage{tikz}
% We need lots of libraries...
\usetikzlibrary{
  arrows,
  calc,
  %fit,
  %patterns,
  %plotmarks,
  shapes.geometric,%
  shapes.misc,
  shapes.symbols,
  %shapes.arrows,
  %shapes.callouts,
  %shapes.multipart,
  %shapes.gates.logic.US,
  %shapes.gates.logic.IEC,
  %circuits.logic.US,
  %circuits.logic.IEC,
  %circuits.logic.CDH,
  %circuits.ee.IEC,
  %datavisualization,
  %datavisualization.formats.functions,
  %er,
  %automata,
  %backgrounds,
  %chains,
  %topaths,
  %trees,
  %petri,
  mindmap,
  %matrix,
  %calendar,
  %folding,
  %fadings,
  %shadings,
  %spy,
  %through,
  %turtle,
  positioning,%
  %scopes,
  %decorations.fractals,
  %decorations.shapes,
  %decorations.text,
  decorations.pathmorphing,
  decorations.pathreplacing,
  %decorations.footprints,
  %decorations.markings,
  %shadows,
  %lindenmayersystems,
  %intersections,
  %fixedpointarithmetic,
  %fpu,
  %svg.path,
  %external,
}%

\tikzset{
  textcircledstyle/.style={shape=circle,draw,inner sep=0.5pt}
}
\newcommand*\tikzcircled[1]{\tikz[baseline=(char.base)]{\node[textcircledstyle, line width=0.8pt] (char) {#1};}}

\newcommand*{\ttbf}[1]{\texttt{\textbf{#1}}}

% Allow font size at arbitrary size
%\usepackage{lmodern}
%\usepackage{fontenc}
%\usepackage{avant}
\usefonttheme{serif}


%\usepackage{mathptmx} % Lucida Bright (SO Version)
%\usepackage[small]{eulervm} % Euler VM%\usepackage{xeCJK}
%\usepackage{euler}
%\usepackage{palatino}
%\usepackage{eulervm}
%\usepackage{bookman}
%\usepackage{chancery}
%\usepackage{helvet}
%\usepackage{utopia}
%\usepackage{charter}
%\usepackage{pifont}

\usetheme{Madrid}
%\usepackage[EU1,T1]{fontenc}

\newcommand{\its}[1]{\textit{\structure{#1}}}
\newcommand{\bfs}[1]{\textbf{\structure{#1}}}

\newcommand{\ita}[1]{\textit{\alert{#1}}}
\newcommand{\bfa}[1]{\textbf{\alert{#1}}}


%\newcommand{\insertdescriptionitem}[1]{\textbf{#1}}

% enumerate item symbol
%\setbeamertemplate{enumerate items}[square]
\setbeamertemplate{enumerate item}[square]
\setbeamertemplate{enumerate subitem}[circle]
\setbeamertemplate{enumerate subsubitem}[ball]

% see above
%\setbeamertemplate{itemize items}[triangle]
\setbeamertemplate{itemize item}[square]
\setbeamertemplate{itemize subitem}[triangle]
\setbeamertemplate{itemize subsubitem}[circle]


% symbol in outline
\setbeamertemplate{sections/subsections in toc}[square]
\setbeamertemplate{frametitle continuation}[from second]

% list the outline at the beginning of each section
\AtBeginSection[]
{
  \begin{frame}<beamer>
    \frametitle{Contents}
    \tableofcontents[currentsection]
  \end{frame}
}

% list the outline at the beginning of each section
\AtBeginSubsection[] % Do nothing for \subsection*
{
  \begin{frame}<beamer>
    \frametitle{Contents}
    \tableofcontents[currentsection,currentsubsection]
  \end{frame}
}

% set color
\setbeamercolor{normal text}{fg=black,bg=white}
\setbeamercolor{alerted text}{fg=red}
\setbeamercolor{example text}{fg=green!50!black}

\definecolor{mycolor}{rgb}{0.0,0.4,0.4} % use structure theme to change
\setbeamercolor{structure}{fg=mycolor}


% A bug, to support bibliography show
\def\newblock{\hskip .11em plus .33em minus .07em}

